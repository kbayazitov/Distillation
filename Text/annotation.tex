\newpage


\begin{abstract}
Исследуется проблема понижения сложности аппроксимирующей модели при переходе к данным домена меньшей мощности. Вводятся понятия учителя, ученика, слабого и сильного доменов. Признаковые описания моделей ученика и учителя принадлежат разным доменам. Мощность одного домена больше мощности другого. Рассматриваются методы, основанные на дистилляции моделей машинного обучения. Вводится предположение, что решение оптимизационной задачи от параметров обеих моделей и доменов повышает качество модели ученика.


\smallskip
\textbf{Ключевые слова}: 
\end{abstract}
