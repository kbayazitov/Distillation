\newpage

\section{Постановка задачи}

\subsection{Базовая постановка задачи дистилляции Хинтона}

Задано множество объектов $\Omega$ и множество целевых переменных $\mathbb{Y}$. Множество $\mathbb{Y}=\{1,...,R\}$ для задачи классификации, где R - число классов, множество $\mathbb{Y}=\mathbb{R}$ для задачи регресии.\\
В постановке Хинтона в качестве модели ученика $\textbf{g}$ рассматривается функция из множества: $$\mathbb{g}=\{\textbf{g}|\textbf{g}=softmax(\textbf{z(x)}/T), \textbf{z}:\mathbb{R}^{n}\rightarrow \mathbb{R}^{R}\}$$\\
В качестве модели учителя $\textbf{f}$ рассматривается функция из множества: $$\mathbb{f}=\{\textbf{g}|\textbf{g}=softmax(\textbf{v(x)}/T), \textbf{v}:\mathbb{R}^{n}\rightarrow \mathbb{R}^{R}\}$$\\
$\textbf{v, z}$ - дифференцируемые параметрические функции заданной структуры, $T$ - параметр температуры со свойствами:\\
1) при $T \rightarrow 0$ получаем вектор, в котором один из классов имеет единичную вероятность;\\ 
2) при $T \rightarrow \infty$ получаем равновероятные классы.\\
Функция потерь $\mathcal{L}$ учитывает перенос информации от модели учителя $\textbf{f}$ к модели ученика $\textbf{g}$ имеет вид $$\mathcal{L}=-\sum\limits_{i=1}^{m}\sum\limits_{r=1}^{R}y_{i}^{r}\log{g(x_{i})}\textbar_{T=1}-\sum\limits_{i=1}^{m}\sum\limits_{r=1}^{R}f(x_{i})\textbar_{T=T_{0}}\log{g(x_{i})}\textbar_{T=T_{0}},$$
где $\cdot\textbar_{T=t}$ означает, что параметр температуры $T$ в предыдущей функции равен $t$.

\subsection{Постановка задачи дистилляции для многодоменной выборки}

Заданы множества объектов $\mathbb{X}, \mathbb{X'}$ - данные первого и второго доменов, и множество целевых переменных $\mathbb{Y}$. Множество $\mathbb{Y}=\{1,...,R\}$ для задачи классификации, где R - число классов, множество $\mathbb{Y}=\mathbb{R}$ для задачи регресии. $\textbf{f,g}$ - модели учителя и ученика соответственно.\\
Рассматриваются отображения $$\varphi: \mathbb{X} \rightarrow \mathbb{X'}, |\mathbb{X'}| \gg |\mathbb{X}|$$ 
$$\textbf{f}: \mathbb{X'} \rightarrow \mathbb{Y}$$
Требуется получить отбражение $$\textbf{g}: \mathbb{X} \rightarrow \mathbb{Y}$$
