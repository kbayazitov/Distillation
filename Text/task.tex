\newpage

\section{Постановка задачи}

\subsection{Базовая постановка задачи дистилляции, предложенной Дж.Хинтоном}

Задана выборка $\textbf{D}=(\textbf{X},\textbf{Y})$, где $\textbf{X} \in \mathbb{X}, \textbf{Y} \in \mathbb{Y}$. Множество $\mathbb{Y}=\{1,...,R\}$ для задачи классификации, где R - число классов, множество $\mathbb{Y}=\mathbb{R}$ для задачи регресии.\\
В качестве модели ученика $\textbf{g}$ рассматривается функция из множества: $$\mathfrak{G}=\{\textbf{g}|\textbf{g}=\text{softmax}(\textbf{z(x)}/T), \textbf{z}:\mathbb{R}^{n}\rightarrow \mathbb{R}^{R}\}$$\\
В качестве модели учителя $\textbf{f}$ рассматривается функция из множества: $$\mathfrak{F}=\{\textbf{f}|\textbf{f}=\text{softmax}(\textbf{v(x)}/T), \textbf{v}:\mathbb{R}^{n}\rightarrow \mathbb{R}^{R}\}$$
$\textbf{v, z}$ - дифференцируемые параметрические функции заданной структуры, $T$ - параметр температуры со свойствами:\\
1) при $T \rightarrow 0$ один из классов имеет единичную вероятность;\\ 
2) при $T \rightarrow \infty$ все классы равновероятны.\\
Функция потерь $\mathcal{L}$, учитывающая модель учителя $\textbf{f}$ при выборе модели ученика $\textbf{g}$, имеет вид: $$\mathcal{L}(\mathbf{w,x,y,f})=-\sum\limits_{i=1}^{m}\sum\limits_{r=1}^{R}y_{i}^{r}\log{g^{r}(x_{i})}\textbar_{T=1}-\sum\limits_{i=1}^{m}\sum\limits_{r=1}^{R}f^{r}(x_{i})\textbar_{T=T_{0}}\log{g^{r}(x_{i})}\textbar_{T=T_{0}},$$
где $\cdot\textbar_{T=t}$ означает, что параметр температуры $T$ в предыдущей функции равен $t$.\\
Получаем оптимизационную задачу:
$$\hat{\mathbf{w}} = \arg\min_{\mathbf{w} \in \mathbb{W}} \mathcal{L}(\mathbf{w,x,y,f}).$$


\subsection{Постановка задачи дистилляции для многодоменной выборки}

Заданы два домена: $$\mathbb{D}_{\text{s}}, \mathbb{D}_{\text{t}}$$ - исходный и целевой наборы данных. Для задачи дистилляции, предложенной Дж.Хинтоном, $\mathbb{D}_{\text{s}}=\mathbb{D}_{\text{t}}$. Предполагается, что числа объектов в доменах не совпадают: $$|\mathbb{X}_{\text{s}}| \gg |\mathbb{X}_{\text{d}}|$$
$\mathbb{Y}$ - множество целевых переменных.\\
Пусть при этом задана модель учителя
$$\textbf{f}: \mathbb{X}_{\text{s}} \rightarrow \mathbb{Y}, \ \text{где } \mathbf{f}-\text{модель учителя}$$
и связь между исходным и целевым доменами:
$$\varphi: \mathbb{X}_{\text{t}} \rightarrow \mathbb{X}_{\text{s}}, \ \text{где } \varphi-\text{необратимое отображение}$$ 
Требуется получить отображение $$\textbf{g}: \mathbb{X}_{\text{t}} \rightarrow \mathbb{Y}, \ \text{где } \mathbf{g}-\text{модель ученика}$$
Функция потерь, учитывающая метки учителя и связь между доменами\\
1) для задачи регрессии:
$$\mathcal{L}(\mathbf{w,x,y,f,\varphi})=\lambda\|\mathbf{y}-\mathbf{g}(\mathbf{x},\mathbf{w})\|_{2}^{2}+(1-\lambda)\|\mathbf{g}(\mathbf{x},\mathbf{w})-(\mathbf{f}\circ \mathbf{\varphi})(\mathbf{x})\|_{2}^{2}$$
2) для задачи классификации:
\[
\begin{aligned}
\mathcal{L}(\mathbf{w,x,y,f,\varphi})=&-\frac{\lambda}{m}\sum\limits_{i=1}^{m}\sum\limits_{r=1}^{R}I[y_{i}=r]\log{g^{r}(x_{i},w)}\\
&-\frac{(1-\lambda)}{m}\sum\limits_{i=1}^{m}\sum\limits_{r=1}^{R}(f\circ \varphi)^{r}(x_{i})\log{g^{r}(x_{i},w)}
\end{aligned}
\]
Получаем оптимизационную задачу:
$$\hat{\mathbf{w}} = \arg\min_{\mathbf{w} \in \mathbb{W}} \mathcal{L}(\mathbf{w,x,y,f,\varphi}).$$
