\newpage

\section{Постановка задачи}

\subsection{Базовая постановка задачи дистилляции}

Задана выборка 
$$\mathfrak{D}=(\mathbf{X},\mathbf{Y}), \mathbf{X} \in \mathbb{X}, \mathbf{Y} \in \mathbb{Y},$$
где множество $\mathbb{Y}=\{1,...,R\}$ для задачи классификации, где $R$ --- число классов, множество $\mathbb{Y}=\mathbb{R}$ для задачи регресии.

Предполагается, что задана обученная модель с большим числом параметров --- модель учителя. Модель учителя $\mathbf{f}$ принадлежит параметрическому семейству функций: $$\mathfrak{F}=\{\mathbf{f}|\mathbf{f}=\text{softmax}(\mathbf{v(x)}/T), \mathbf{v}:\mathbb{R}^{n}\rightarrow \mathbb{R}^{R}\}.$$

Требуется обучить модель ученика с меньшим числом параметров с учетом ответов учителя. Модель ученика $\mathbf{g}$ принадлежит параметрическому семейству функций: $$\mathfrak{G}=\{\mathbf{g}|\mathbf{g}=\text{softmax}(\mathbf{z(x)}/T), \mathbf{z}:\mathbb{R}^{n}\rightarrow \mathbb{R}^{R}\},$$
где $\mathbf{v, z}$ --- дифференцируемые параметрические функции заданной структуры, $T$ --- параметр температуры со свойствами:
\begin{enumerate}
    \item при $T \rightarrow 0$ один из классов имеет единичную вероятность;
    \item при $T \rightarrow \infty$ все классы равновероятны.
\end{enumerate}

Функция потерь $\mathcal{L}$, учитывающая модель учителя $\mathbf{f}$ при выборе модели ученика $\mathbf{g}$, имеет вид:
\[
\begin{aligned}
    \mathcal{L}(\mathbf{w,X,Y,f})=-\sum\limits_{i=1}^{m}\sum\limits_{r=1}^{R}y_{i}^{r}\log{g^{r}(x_{i})}\bigr|_{T=1}-\sum\limits_{i=1}^{m}\sum\limits_{r=1}^{R}f^{r}(x_{i})\bigr|_{T=T_{0}}\log{g^{r}(x_{i})}\bigr|_{T=T_{0}},
\end{aligned}
\]
где $\cdot\bigr|_{T=t}$ означает, что параметр температуры $T$ в предыдущей функции равен $t$.

Получаем оптимизационную задачу:
$$\hat{\mathbf{w}} = \arg\min_{\mathbf{w} \in \mathbb{W}} \mathcal{L}(\mathbf{w,X,Y,f}).$$


\subsection{Постановка задачи дистилляции для многодоменной выборки}

Заданы две выборки:
$$\mathfrak{D}_{\text{s}}=(\mathbf{X}_{\text{s}},\mathbf{Y}_{\text{s}}), \quad \mathbf{X}_{\text{s}} \in \mathbb{X}_{\text{s}}, \mathbf{Y}_{\text{s}} \in \mathbb{Y}$$
$$\mathfrak{D}_{\text{t}}=(\mathbf{X}_{\text{t}},\mathbf{Y}_{\text{t}}), \quad \mathbf{X}_{\text{t}} \in \mathbb{X}_{\text{t}}, \mathbf{Y}_{\text{t}} \in \mathbb{Y},$$
где $\mathfrak{D}_{\text{s}}, \mathfrak{D}_{\text{t}}$ --- исходный и целевой наборы данных. Для задачи дистилляции, предложенной Джеффри Хинтоном, $\mathfrak{D}_{\text{s}}=\mathfrak{D}_{\text{t}}$.

Предполагается, что число объектов в выборках не совпадают:
$$|\mathbb{X}_{\text{s}}| \gg |\mathbb{X}_{\text{t}}|$$

Пусть при этом задана модель учителя на выборке большей мощности:
$$\mathbf{f}: \mathbb{X}_{\text{s}} \rightarrow \mathbb{Y},$$
где $\mathbf{f}$ --- модель учителя.

Задана связь между исходной и целевой выборками:
$$\varphi: \mathbb{X}_{\text{t}} \rightarrow \mathbb{X}_{\text{s}},$$
где $\varphi$ --- отображение, для которого существование обратного не рассматривается.

Требуется получить модель ученика для малоресурсной выборки:
$$\mathbf{g}: \mathbb{X}_{\text{t}} \rightarrow \mathbb{Y},$$
где $\mathbf{g}$ --- модель ученика.

В работе рассматривается функция потерь, учитывающая метки учителя и связь между доменами:
\begin{enumerate}
    \item для задачи регрессии:
    \[
    \begin{aligned}
    \mathcal{L}(\mathbf{w,X,Y,f,\varphi})=\lambda\|\mathbf{y}-\mathbf{g}(\mathbf{x},\mathbf{w})\|_{2}^{2}+(1-\lambda)\|\mathbf{g}(\mathbf{x},\mathbf{w})-(\mathbf{f}\circ \mathbf{\varphi})(\mathbf{x})\|_{2}^{2};
    \end{aligned}
    \]
    \item для задачи классификации:
    \[
    \begin{aligned}
    \mathcal{L}(\mathbf{w,X,Y,f,\varphi})=&-\lambda\sum\limits_{i=1}^{m}\sum\limits_{r=1}^{R}I[y_{i}=r]\log{g^{r}(\mathbf{x}_{i},\mathbf{w})}\\
    &-(1-\lambda)\sum\limits_{i=1}^{m}\sum\limits_{r=1}^{R}(f\circ \varphi)^{r}(\mathbf{x}_{i})\log{g^{r}(\mathbf{x}_{i},\mathbf{w})},
    \end{aligned}
    \]
    где $\lambda$ --- метапараметр, задающий вес дистилляции.
\end{enumerate}

Получаем оптимизационную задачу:
$$\hat{\mathbf{w}} = \arg\min_{\mathbf{w} \in \mathbb{W}} \mathcal{L}(\mathbf{w,X,Y,f,\varphi}).$$
