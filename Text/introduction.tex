\newpage


\section{Введение}
Сбор и обработка наборов данных для каждой новой задачи и области являются чрезвычайно дорогими и трудоемкими процессами, и не всегда могут быть доступны достаточные данные для обучения. Цель данной работы заключается в понижении сложности модели машинного обучения при переходе к домену меньшей мощности. Для этого предлагается использовать два основных метода - дистилляция моделей и доменная адаптация.\\
Дистилляция моделей машинного обучения использует метки модели с большим числом параметров для обучения модели с меньшим числом параметров. В~\cite{Hinton2015} рассматривается метод дистилляции, предложенной Дж.Хинтоном, с учетом меток учителя при помощи функции \text{softmax} с параметром температуры, а в~\cite{Vapnik2016} рассматривается объединение методов дистилляци, предложенной Дж.Хинтоном, и привилегированной информации, предложенной В.Вапником, в обобщенную дистилляцию. Дистиляцция моделей используется в широком классе задач. В~\cite{MDASR} рассматривается метод дистилляции моделей для задачи распознавания речи.\\
Часто выборки могут состоять из объектов, которые можно разделить на домены. К примеру, можно составить отображение из множества реальных фотографий малой мощности во множество сгенерированных движком изображений, мощность которого естественно больше. Для задачи дистилляции, предложенной Дж.Хинтоном, исходный и целевой домены равны. Различные постановки задач доменной адаптации описываются в~\cite{DeepvisDA}, встречаются постановки с частично размеченным целевым доменом и неразмеченным вовсе. Таким образом, доменная адаптация использует размеченные данные нескольких исходных доменов для выполнения новых задач в целевом домене.\\
Типичной задачей дистилляции моделей на многодоменных выборках является задача машинного перевода текстов, описанная в~\cite{KimRush2016}.\\
В качестве экспериментальных данных используются реальные данные и синтетическая выборка. В качестве реальных данных рассматривается выборка \text{FashionMnist}~\cite{FMNIST}, состоящая из изображений одежды, для которой требуется решить задачу классификации на 10 типов одежды.

