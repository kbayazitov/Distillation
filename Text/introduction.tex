\newpage


\section{Введение}
Доменная адаптация использует размеченные данные нескольких доменов для выполнения новых задач в целевом домене.\\
Исходный и целевой домены могут содержать изображения, тогда расхождение признаковых описаний может быть вызвано разными сенсорными устройствами и разными стилями изображений (рисунки и фотографии).\\
Дистилляция моделей машинного обучения использует метки модели с большим числом параметров для обучения модели с меньшим числом параметров.\\

\section{Анализ литературы}

В~\cite{Hinton2015} рассматривается метод дистилляции с учетом меток учителя при помощи функции softmax с параметром температуры.\\
В~\cite{Vapnik2016} рассматривается объединение методов дистилляции Хинтона и привилегированной информации Вапника в обобщенную дистилляцию.\\
В~\cite{KimRush2016} рассматривается метод дистилляции моделей для задачи перевода текстов.\\
В~\cite{MDASR} рассматривается метод дистилляии моделей для задачи распознавания речи.\\
В~\cite{DeepvisDA} рассматривается задача машинного обучения при наличии исходного и целевых доменов.\\
В~\cite{FMNIST} приводится описание выборки, на которой проводятся все эксперименты.
